\documentclass[stu, 12pt, letterpaper, donotrepeattitle, floatsintext, natbib]{apa7}
\usepackage[utf8]{inputenc}
\usepackage{comment}
\usepackage{marvosym}
\usepackage{graphicx}
\usepackage{float}
\usepackage[normalem]{ulem}
\usepackage[spanish]{babel} 

% Required package
\usepackage{amsmath}
\usepackage{xcolor}

\selectlanguage{spanish}
\useunder{\uline}{\ul}{}
\newcommand{\myparagraph}[1]{\paragraph{#1}\mbox{}\\}

% Portada
\thispagestyle{empty}
\title{\Large Calculo Diferencial e Integral}
\author{Cristian Herrera}
\affiliation{Instituto Superior Tecnológico Tena}
\course{Calculo Diferencial e Integral}
\professor{Ing. Libinton Lara}
\duedate{2023 - IIP}
\begin{document}
\maketitle


% Índices
\pagenumbering{roman}
    % Contenido
\renewcommand\contentsname{\largeÍndice}
\tableofcontents
\setcounter{tocdepth}{2}
\newpage
    % Fíguras
\renewcommand{\listfigurename}{\largeÍndice de fíguras}
\listoffigures
\newpage
    % Tablas
\renewcommand{\listtablename}{\largeÍndice de tablas}
\listoftables
\newpage

% Cuerpo
\pagenumbering{arabic}

\section{Actividades en Clase}
\subsection{Matrices y Determinantes} 

\subsubsection{Que es una matriz?}
Conjunto bidimencional de numeros o simbolos que sirven para describir y resolver sistemas de ecuaciones lineales.
\begin{table}
    \centering
    \begin{tabular}{|c|c|}
        \hline
        Sistema de Ecuaciones & En Forma de Matriz \\ \hline
        & $\quad x\quad y\quad i$ \\ 
        $ f(x) =  \begin{cases} x + 3y = 7 \\ 5x - y = 3  \end{cases} $ 
        & 
        $A = \begin{pmatrix}
            1 & 3 & 7 \\
            5 & -1 & 3
        \end{pmatrix}_{2\times3}$ \\[1cm]\hline
        
        & $\quad\quad\quad x\quad i\quad $ \\
        $2x=4$ & $A = \begin{pmatrix} 2 & 4 \end{pmatrix}$ \\[0.8cm]\hline
        
        & $\quad\quad x\quad\quad y\quad\quad z\quad\quad i$ \\
        $ f(x)=\begin{cases} 3x-5y+8z=10 \\ 2y-7z=-15 \end{cases} $ & 
        $C=\begin{pmatrix}
            3 & -5 & 8 & 10 \\ 
            0 & 2 & -2 & -15
        \end{pmatrix}$ \\[1cm]\hline
    \end{tabular}
\end{table}

\subsubsection{Dimensiones de una matriz}
$$
A=\begin{pmatrix}
a_{11} & a_{12} & a_{13} & a_{14}\\
a_{21} & a_{22} & a_{23} & a_{24}\\
\end{pmatrix}_{M\times N}
$$\\
\begin{small}
\begin{center}
$M= $Filas, $N= $Columnas
\end{center}
\end{small}

Un elemento de una matriz se representa: {\Large $a_{ij}$}, donde {\Large $i$} representa las filas, y {\Large $j$} las columnas\\

\subsubsection{Tipos de Matrices}
\paragraph{Matriz Fila (Vector Fila)}
Matriz formada por una sola fila.
$$A=\begin{pmatrix} a_{11} & a_{12} & a_{13} & .... & a_{1n} \end{pmatrix}_{1\times n}$$
\paragraph{Matriz Columna (Vector Columna)}
Matriz formada por una sola columna.
$$A=\begin{pmatrix} a_{11} \\ a_{21} \\ a_{31} \\ ... \\ ... \\ ... \\ a_{m1}\end{pmatrix}_{m\times1}$$
\paragraph{Matriz Nula}
Todos sus elementos son nulos
$$A=\begin{pmatrix} 0 & 0 & 0 & 0 \\ 0 & 0 & 0 & 0 \end{pmatrix}_{2\times4}$$
\paragraph{Matriz Cuadrada}
Tiene el mismo numero de filas y de columnas.

\begin{table}
\centering
\begin{tabular}{cc}
$
A = \begin{pmatrix}
    \color{red} a_{11} & a_{12} & a_{13} \\
    a_{21} & \color{red} a_{22} & a_{23} \\
    a_{31} & a_{32} & \color{red} a_{33} \\
\end{pmatrix}_{3\times3}
$
&
$
B = \begin{pmatrix}
    a_{11} & a_{12} & \color{blue} a_{13} \\
    a_{21} & \color{blue} a_{22} & a_{23} \\
    \color{blue} a_{31} & a_{32} & a_{33} \\
\end{pmatrix}_{3\times3}
$
\end{tabular}
\end{table}
\subparagraph{Elementos básicos de una Matriz Cuadrada}
\myparagraph{Diagonal Mayor}
Formada por los elementos $\color{red} a_{11}$, $\color{red} a_{22}$, $\color{red} a_{33}$, $\color{red} \ldots$, $\color{red} a_{nn}$.
\myparagraph{Diagonal Menor}
Formada por los elementos {\Large $\color{blue} a_{ij}$} donde {\normalsize $\color{blue} i+j = n+1$}
\myparagraph{Traza}
Suma de los elementos de la diagonal principal.

\begin{table}
\centering
\begin{tabular}{cc}
$
A=\begin{pmatrix}
\color{red} -2 & 6 & 4 \\
8 & \color{red} 12 & -9 \\
15 & -6 & \color{red} -4
\end{pmatrix}
$ & traza: $-2 +12 -4 = 6$
\end{tabular}
\end{table}

\subparagraph{Tipos básicos de matrices cuadradas}
\myparagraph{Triangular Superior}
Todos los elementos son ceros debajo de la diagonal principal.
$$
A=\begin{pmatrix}
\color{gray} -2 & \color{gray} 6 & \color{gray} 4 \\
0 & \color{gray} 12 & \color{gray} -9 \\
0 & 0 & \color{gray} -4
\end{pmatrix}
$$
\myparagraph{Triangular Inferior}
Encima de la diagonal principal todos son ceros.
$$
A=\begin{pmatrix}
\color{gray} -2 & 0 & 0 \\
\color{gray} 8 & \color{gray} 12 & 0 \\
\color{gray} 15 & \color{gray} -6 & \color{gray} -4
\end{pmatrix}
$$
\myparagraph{Matriz Diagonal}
Los elementos que no están en la diagonal principal son ceros, esta también es una matriz triangular superior y triangular inferior.
$$
A=\begin{pmatrix}
\color{gray} -2 & \textbf{0} & \textbf{0} \\
\textbf{0} & \color{gray} 12 & \textbf{0} \\
\textbf{0} & \textbf{0} & \color{gray} -4
\end{pmatrix}
$$
\myparagraph{Matriz identidad o unidad}
Es una matriz diagonal, en la que todos los elementos de la diagonal principal son 1.
\[
A=\begin{bmatrix}
\textbf{1} & 0 & 0 \\ 
0 & \textbf{1} & 0 \\
0 & 0 & \textbf{1}
\end{bmatrix}
\]
\myparagraph{Matriz Escalar}
Es una matriz diagonal, en la que todos los elementos de la diagonal principal son iguales.
\[
A=\begin{bmatrix}
\textbf{5} & \color{gray} 0 & \color{gray} 0 \\ 
 \color{gray} 0 & \textbf{5} & \color{gray} 0 \\
 \color{gray}0 &  \color{gray} 0 & \textbf{5}
\end{bmatrix}
\]
\subsubsection{Suma de Matrices}
Se puede dar entre matrices del mismo orden, para realizar la suma se debe sumar cada elemento de una posición con otro elemento de una matriz de la misma posición.

\begin{table}
\centering
\begin{tabular}{c}
$
A=\begin{pmatrix}
a_{11} & a_{12} & a_{13} & a_{14}\\
a_{21} & a_{22} & a_{23} & a_{24}\\
\end{pmatrix}_{2\times4}
$  $
B=\begin{pmatrix}
b_{11} & b_{12} & b_{13} & b_{14}\\
b_{21} & b_{22} & b_{23} & b_{24}\\
\end{pmatrix}_{2\times4}
$\\ \\ $
A+B=\begin{pmatrix}
(a_{11} + b_{11})_{11} & (a_{12} + b_{12})_{12} & (a_{13} + b_{13})_{13} & (a_{14} + b_{14})_{14}\\
(a_{21} + b_{21})_{21} & (a_{22} + b_{22})_{22} & (a_{23} + b_{23})_{23} & (a_{24} + b_{24})_{24}\\
\end{pmatrix}_{2\times4}
$

\end{tabular}
\end{table}

\subsubsection{Resta de Matrices}
Se puede dar entre matrices del mismo orden, para realizar la resta se debe restar cada elemento de una posición con otro elemento de una matriz de la misma posición.

\begin{table}[H]
\centering
\begin{tabular}{c}
$
A=\begin{pmatrix}
a_{11} & a_{12} & a_{13} & a_{14}\\
a_{21} & a_{22} & a_{23} & a_{24}\\
\end{pmatrix}_{2\times4}
$  $
B=\begin{pmatrix}
b_{11} & b_{12} & b_{13} & b_{14}\\
b_{21} & b_{22} & b_{23} & b_{24}\\
\end{pmatrix}_{2\times4}
$\\ \\ $
A-B=\begin{pmatrix}
(a_{11} - b_{11})_{11} & (a_{12} - b_{12})_{12} & (a_{13} - b_{13})_{13} & (a_{14} - b_{14})_{14}\\
(a_{21} - b_{21})_{21} & (a_{22} - b_{22})_{22} & (a_{23} - b_{23})_{23} & (a_{24} - b_{24})_{24}\\
\end{pmatrix}_{2\times4}
$

\end{tabular}
\end{table}



\subsubsection{Producto De Una Matriz Por Un Escalar/Real}
La multiplicación de una matriz por un escalar se realiza multiplicando el escalar por cada elemento de la matriz.

\begin{table}[H]
\centering
\begin{tabular}{ccc}
$
\dfrac{3}{2} \times \begin{pmatrix}
2 & -4\\
1 & -8\\
\end{pmatrix}_{2\times2}
$  & $=$ & $
\begin{pmatrix}
3 & -6 \\
\frac{3}{2} & -12\\
\end{pmatrix}_{2\times2}
$\\ 

\end{tabular}
\end{table}


\subsubsection{Multiplicación de Matrices}
Es posible si el numero de \textbf{columnas} de la primera matriz es el mismo al numero de \textbf{filas} de la segunda matriz.
Para multiplicar dos matrices cuadradas, calcula el producto de filas de la primera matriz con columnas de la segunda matriz y suma los resultados.
$$
A=\begin{pmatrix}
1 & 3 \\ 4 & 5
\end{pmatrix}_{2\times2}
\qquad
B=\begin{pmatrix}
6 & 7 \\ 8 & 9
\end{pmatrix}_{2\times2}
$$\\

¿Es posible multiplicar la matriz A por la matriz B?
$${\textit{2}\times\textbf{2}} \qquad {\textbf{2}\times\textit{2}}$$\\

Si es posible ya que el numero de columnas de la primera fila es el mismo que el numero de filas de la segunda, $\textbf{2}$ = $\textbf{2}$, ademas el resultado de $A\times B$ sera igual a una matriz de $\textit{2} \times \textit{2}$.
\begin{center}
\includegraphics[scale=0.5]{matrix.png} 
\end{center}

\[
\begin{bmatrix}
   \color{blue} 2 & \color{blue} 3 \\
    \color{red} 4 & \color{red} 5
\end{bmatrix}
\times
\begin{bmatrix}
    \color{green} 6 & \color{violet} 7 \\
    \color{green} 8 & \color{violet} 9
\end{bmatrix}
=
\begin{bmatrix}
    (\color{blue} 2 \textcolor{black}{\,\cdot\,} \color{green} 6 \textcolor{black}{\,+\,} \color{blue} 3 \textcolor{black}{\,\cdot\,} \textcolor{green}{8} \textcolor{black}{)} 
    &
     (\color{blue} 2 \textcolor{black}{\,\cdot\,} \color{violet} 7 \textcolor{black}{\,+\,} \color{blue} 3 \textcolor{black}{\,\cdot\,} \textcolor{violet}{9} \textcolor{black}{)} \\
          
    (\color{red} 4 \textcolor{black}{\,\cdot\,} \color{green} 6 \textcolor{black}{\,+\,} \color{red} 5 \textcolor{black}{\,\cdot\,} \textcolor{green}{8} \textcolor{black}{)}
    &    
    (\color{red} 4 \textcolor{black}{\,\cdot\,} \color{violet} 7 \textcolor{black}{\,+\,} \color{red} 5 \textcolor{black}{\,\cdot\,} \textcolor{violet}{9} \textcolor{black}{)}
\end{bmatrix}
= AB=
\begin{bmatrix}
    36 & 41 \\
    68 & 77
\end{bmatrix}_{2\times2}
\]



\subsubsection{Determinantes}
El determinante de una matriz solo existe en matrices cuadradas.
\paragraph{Determinante 2x2}
Se calcula restando el producto de los números que conforman la diagonal secundaria al producto de los números que conforman la diagonal principal. 
\[
A=\begin{pmatrix}
b & c \\ d & e
\end{pmatrix}
\qquad
\qquad
|A|= (be) - (cd)
\]
\paragraph{Determinante 3x3 - Regla de Sarrus}
Las primeras columnas o columnas de la matriz se repiten al final y se realiza una multiplicación de diagonales para obtener el determinante.\\
\begin{table}
\begin{tabular}{cc}

Primera forma & Segunda forma \\
\includegraphics[scale=0.25]{sarrus}
&
\includegraphics[scale=0.5]{sarrus2}

\end{tabular}
\end{table}
\newpage
% Referencias
\renewcommand\refname{\large\textbf{Referencias}}
\bibliography{mibibliografia}

\end{document}